\documentclass[twocolumn,12pt]{article} 
\usepackage{luatexja}           % 日本語対応
\usepackage{luatexja-fontspec}  % 日本語 + fontspec 連携
\usepackage{fontspec}          

\setmainfont{Times New Roman}        % 英文フォント(エラー回避)
\setmainjfont{Hiragino Mincho ProN}  % 日本語フォント(macOS用)

\title{スマートフォンを用いた持参薬識別支援システムの開発と有用性の検討}
\author{久田 成真}
\date{}

\begin{document}

\maketitle
\section{序論}

入院患者が常用薬を持参することは一般的だが、特に多剤併用の高齢患者では形状や色が似た薬が多く、迅速な識別が難しい。薬剤師は病室で直接確認するものの、時間と労力がかかり、誤認リスクもある。資料や外観に頼るため、患者の誤申告やラベル欠損があると判別はさらに困難になる。

そこで、スマートフォンを用いた薬剤識別支援システムが求められる。本研究は、病室で薬剤師が手軽に薬を特定できるスマホベースの検知技術の有用性と実装可能性を検討し、業務効率向上と誤投与防止を目指す。


\section{関連研究}


Dangらの研究\cite{Dang_2024}では、視覚障害者がスマホで錠剤を識別できるアプリを開発し、YOLOv8 を錠剤検出向けにカスタマイズして32カテゴリ・mAP 99.5\% を達成した。学習には米国Pillboxデータセット(8693枚)を用い、錠剤の形状や刻印を高精度で捉えられるよう畳み込み層を最適化している。
Kwonらの研究では、\cite{chemosensors10010004}少量データで性能を高めるため Mask R-CNN による二段階検出を提案。① 複数錠剤画像で錠剤領域を検出し、② 単一錠剤画像で種類を判定する。自動ラベリングと露光・回転のデータ拡張、後処理(領域拡張や段階的回転検出など)で精度と正答率を向上させ、YOLOv3 を上回った。これにより自動薬剤払い出し機の検査効率向上が期待され、今後は多環境での検証や Transformer 技術の導入が課題となる。


\bibliographystyle{plain}  % スタイル(plain, unsrt, alpha など)
\bibliography{refs}        % refs.bib を参照(拡張子は書かない)


\end{document}
