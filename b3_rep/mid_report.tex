\documentclass[report]{jlreq}
\usepackage{multicol}
\usepackage{setspace}
\usepackage{graphicx}
\renewcommand{\baselinestretch}{.856}
\begin{document}


\begin{multicols}{2}

{\LARGE
    卒業研究タイトル案:薬局のDX

\vspace{3mm}
}

\vspace{-2mm}
\begin{flushright}
電子商取引研究室\hspace{10pt}久田成真
\end{flushright}
\vspace{3mm}
\end{multicols}


\begin{multicols}{2}
\section{序論}
薬剤師は患者に薬を処方するという大事な仕事がある一方、負担も大きい。

・持ち込み薬剤の分類

・医薬品の添付文章からの副作用などの自動取得

などを実現することで、薬剤師の負担を減らせることができるかもしれない。
\section{研究内容}

\includegraphics[width=5cm]{plan1.png}

\includegraphics[width=5cm]{plan2.png}

\section{今後の計画}

\includegraphics[width=5cm]{plan3.png}

\bibliographystyle{jplain}
\begin{thebibliography}{1}
    \item{三浦康秀2010電子カルテからの副作用関係の自動抽出,
,
{三浦康秀 and 荒牧英治 and 大熊智子 and 外池昌嗣 and 杉原大悟 and 増市博 and 大江和彦},
{言語処理学会第 16 回年次大会},
{78--81},
{2010}
}
\end{thebibliography}

\end{multicols}
\end{document}




