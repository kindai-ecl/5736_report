\documentclass[twocolumn,12pt]{article} 
\usepackage{luatexja}           % 日本語対応
\usepackage{luatexja-fontspec}  % 日本語 + fontspec 連携
\usepackage{fontspec}          

\setmainfont{Times New Roman}        % 英文フォント(エラー回避)
\setmainjfont{Hiragino Mincho ProN}  % 日本語フォント(macOS用)

\title{スマートフォンを用いた持参薬識別支援システムの開発と有用性の検討}
\author{久田 成真}
\date{}

\begin{document}

\maketitle
\section{序論}

現代の医療現場において、入院患者が自身の常用薬を持参することは一般的な光景である。しかし、これに伴い臨床現場では、薬剤の正確な識別という重要な課題が浮上する。特に多剤併用を行っている高齢患者においては、薬剤の種類が多岐にわたり、形状や色が似通っているため、迅速かつ正確な判別が困難となる場合が多い。

通常、薬剤師は患者の病室まで出向き、直接確認作業を行うが、このプロセスは時間と労力を要するだけでなく、誤認識のリスクも内在している。さらに、薬剤師が持参した資料や外観のみの判断に頼ることが多く、患者の誤った申告やラベルの欠損などがあれば、正確な判別がさらに困難になる可能性がある。

こうした課題を解決するため、スマートフォンなどのモバイルデバイスを活用した薬剤識別支援システムの開発が求められている。本研究では、薬剤師が病室で迅速かつ確実に薬剤を識別できるよう、スマートフォンベースの検知技術の有用性とその実装可能性について検討する。これにより、臨床現場での業務効率向上と薬剤誤投与の防止に寄与することを目的とする。

\section{関連研究}


Dangらの研究\cite{Dang_2024}では、視覚障害のある人々がスマートフォンのカメラを使用して錠剤を識別できるようにアプリを作っている。
その中で、薬の錠剤を識別するシステムを設計している。
このアプリケーションでは、YOLO(You Only Look Once)というオブジェクト検出モデルの最新版であるYOLOv8フレームワークが活用されている。YOLOは、入力画像を処理して特徴を抽出し、それをグリッドに分割することで、各セルでオブジェクトの有無を効率的に予測することができる。

モデルのトレーニングには、アメリカ国立医学図書館が管理するPillboxデータセットが使用されている。このデータセットには、米国内で販売されている錠剤の写真8,693枚が含まれており、それぞれに形状、色、刻印、サイズ、有効成分、製造元といった詳細な薬剤情報が付随している。こうした情報は、さまざまな種類の錠剤の形状や識別の違いを認識するために不可欠である。

さらに、YOLOv8のアーキテクチャは錠剤検出に特化するように変更されており、錠剤固有の特徴をより正確に捉えられるようカスタマイズされている。特に、畳み込み層は錠剤の形状や刻印の微細な違いを検出できるようにファインチューニングされており、これがバウンディングボックスの精度や錠剤の分類精度を高めている。現在、このモデルは32種類の異なる錠剤カテゴリを予測可能であり、平均平均精度(mAP)99.5\%という高い精度で錠剤の検出と識別を実現している。

Kwonらの研究では、\cite{chemosensors10010004}限られた学習データで医薬品検査のための錠剤検出性能を向上させるディープラーニングアルゴリズムを提案している。複数の錠剤が写った画像において個々の錠剤を検出する際、錠剤の種類が増加すると画像中の錠剤の組み合わせが指数関数的に増え、学習データの取得およびラベル付けの作業が複雑になる。この問題に対処するため、本研究では単一の錠剤画像のみを学習データとして使用する手法を提案している。

提案手法では、Mask R-CNNに基づく2段階の検出プロセスを採用している。第1段階では、錠剤の種類に関係なく画像中の錠剤の数と領域(エリア)を検出する。この学習には、複数の錠剤が写った画像を使用している。第2段階では、第1段階で検出された錠剤を背景から分離し、対応する錠剤の種類を検出する。この段階では、単一錠剤画像を学習に用いている。

また、第1段階で得られた錠剤領域検出モデルを用いて、単一錠剤画像に対するデータラベリングを自動化し、JSONファイルを生成する手法も提案している。これにより、学習データの準備にかかる時間と労力の削減が可能となる。

さらに、学習中のデータ不足を補うために、露光や回転によるデータ拡張を行っている。加えて、隣接する錠剤による誤検出や非検出、特定の角度での検出失敗といった問題を解決するために、後処理アルゴリズムを適用している。これには、領域の拡張、複数の検出結果からの最大領域の選択、非検出時の段階的な回転検出などが含まれる。

実験の結果、提案手法は限定的な画像枚数と小規模なデータセットにもかかわらず、既存のアルゴリズムと比較して高い検出性能を示した。特に、後処理アルゴリズムの適用によって、精度および正答率が向上した。また、白色錠剤の側面画像に対するデータ収集条件の追加も検出性能の向上に寄与している。YOLOv3との比較では、提案手法の方が正答率が高い結果を得ている。

この手法は、自動薬剤払い出し機などの自動装置の性能向上に貢献し、払い出された錠剤の検査において生産性の損失やヒューマンエラーを最小限に抑えることが期待される。今後の課題としては、さまざまな環境下での実験の実施や、最新のTransformerベース技術の応用が挙げられる。



\bibliographystyle{plain}  % スタイル(plain, unsrt, alpha など)
\bibliography{refs}        % refs.bib を参照(拡張子は書かない)


\end{document}
